\documentclass[a4paper,10pt]{article}
%\documentclass[a4paper,10pt]{scrartcl}

\usepackage[utf8]{inputenc}
\usepackage[total={7.5in, 11in}]{geometry}
\usepackage{amsmath}
\usepackage{amsfonts}
\usepackage{amssymb}

\title{Formule GAL}
\author{Nicolò Luigi Allegris}
\date{}

\pdfinfo{%
  /Title    (Formule GAL)
  /Author   (Nicolò Luigi Allegris)
  /Creator  (Nicolò Luigi Allegris)
  /Producer (Nicolò Luigi Allegris)
  /Subject  (Geometria e Algebra Lineare)
  /Keywords ()
}

\begin{document}
\maketitle

\paragraph{Matrici Simili} Due matrici $A, A' \in {\cal{M}}_{n,n}(\mathbb{K})$
si chiamano simili se $\exists P \in {\cal{M}}_{n.n}(\mathbb{K})$ invertibile,
t.c. $A' = P \cdot A \cdot P^{-1}$.
\subparagraph{Prop} Matrici simili hanno lo stesso determinante.

\paragraph{Endomorfismo} Un endomorfismo di $U$ è un applicazione lineare $f: U
\rightarrow U$.

\subparagraph{Determinante di un endomorfismo} Sia $f: U \rightarrow U$
endomorfismo, Il determinante di $f$ è dato da
$det\left({\cal{M}}_{\beta}^{\beta}\left(f\right)\right)$ dove $\beta$ è una
base di $U$.

\subparagraph{Determinante di una composta} Siano $f,g: U \rightarrow U$
endomorfismi, allora $det\left(f \circ g \right) = det\left(f \right) \cdot
det\left(g \right)$

\subparagraph{Prop} $f: U \rightarrow U$ isomorfismo $\Leftrightarrow
det\left(f\right) \neq 0$

\paragraph{Diagonalizzazione} Dato un endomorfismo $f: V \rightarrow V$ bisogna
trovare una base $\beta$ di $V$ adattata a $f$, cioè tale che la matrice di $f$
rispetto a $\beta$ sia diagonale.
$${\cal M}_\beta^\beta\left(f\right) =
\begin{pmatrix}
\alpha_1 & \cdots & 0 \\
\vdots & \ddots & \vdots \\
0 & \ldots & \alpha_n
\end{pmatrix}$$
rendendo in questo modo il calcolo di $f$ con prodotto matriciale della sua
matrice associata facile. \\
Una matrice è diagonalizzabile se e solo se è simile a una matrice diagonale.

\paragraph{Autovalori e Autovettori} Sia $f: U \rightarrow U$ un endomorfismo.
\begin{enumerate}
  \item Uno scalare $k \in \mathbb{K}$ si chiama autovalore di $f$ se
$\exists v \in V$ non nullo t.c. $f\left(v\right) = \alpha v$
  \item Un vettore $V \in V$ si chiama autovettore di $f$ se
$v \neq 0$ e $\exists \alpha \in \mathbb{K}$ t.c. $f\left(v\right) = \alpha v$
\end{enumerate}

\subparagraph{Oss} Diagonalizzare $f$ equivale a trovare una base formata di
autovettori.

\subparagraph{Oss} $0$ autovalore $\Leftrightarrow ker\left(f\right) \neq 0
\Leftrightarrow f$ non è un isomorfismo.

\paragraph{Teorema} Sia $f: V \rightarrow V$ un endomorfismo. Siano $v_1, ...,
v_k \in V$ autovettori di $f$ corrispondenti ad autovalori distinti $\alpha_1,
..., \alpha_k$. Allora $v_1, ..., v_k$ sono linearmente indipendenti.
\end{document}
