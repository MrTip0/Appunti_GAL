\documentclass[a4paper,10pt]{article}
%\documentclass[a4paper,10pt]{scrartcl}

\usepackage[utf8]{inputenc}
\usepackage[total={7.5in, 10in}]{geometry}
\usepackage{amsmath}
\usepackage{amsfonts}
\usepackage{amssymb}

\title{Appunti GAL}
\author{Nicolò Luigi Allegris}
\date{}

% \pdfinfo{%
%   /Title    (Appunti GAL)
%   /Author   (Nicolò Luigi Allegris)
%   /Creator  (Nicolò Luigi Allegris)
%   /Producer (Nicolò Luigi Allegris)
%   /Subject  (Geometria e Algebra Lineare)
%   /Keywords ()
% }

\begin{document}
\maketitle

\paragraph{Matrici Simili} Due matrici $A, A' \in {\cal{M}}_{n,n}(\mathbb{K})$
si chiamano simili se $\exists P \in {\cal{M}}_{n.n}(\mathbb{K})$ invertibile,
t.c. $A' = P \cdot A \cdot P^{-1}$.
\subparagraph{Prop} Matrici simili hanno lo stesso determinante.

\paragraph{Endomorfismo} Un endomorfismo di $U$ è un applicazione lineare $f: U
\rightarrow U$.

\subparagraph{Determinante di un endomorfismo} Sia $f: U \rightarrow U$
endomorfismo, Il determinante di $f$ è dato da
$det\left({\cal{M}}_{\beta}^{\beta}\left(f\right)\right)$ dove $\beta$ è una
base di $U$.

\subparagraph{Determinante di una composta} Siano $f,g: U \rightarrow U$
endomorfismi, allora $det\left(f \circ g \right) = det\left(f \right) \cdot
det\left(g \right)$

\subparagraph{Prop} $f: U \rightarrow U$ isomorfismo $\Leftrightarrow
det\left(f\right) \neq 0$

\section{Diagonalizzazione}
Dato un endomorfismo $f: V \rightarrow V$ bisogna trovare una base $\beta$ di
$V$ adattata a $f$, cioè tale che la matrice di $f$ rispetto a $\beta$ sia
diagonale.
\[{\cal M}_\beta^\beta\left(f\right) =
\begin{pmatrix}
\alpha_1 & \cdots & 0 \\
\vdots & \ddots & \vdots \\
0 & \ldots & \alpha_n
\end{pmatrix}\]
rendendo in questo modo il calcolo di $f$ con prodotto matriciale della sua
matrice associata facile. \\
Una matrice è diagonalizzabile se e solo se è simile a una matrice diagonale.

\paragraph{Autovalori e Autovettori} Sia $f: U \rightarrow U$ un endomorfismo.
\begin{enumerate}
  \item Uno scalare $k \in \mathbb{K}$ si chiama autovalore di $f$ se
$\exists v \in V$ non nullo t.c. $f\left(v\right) = \alpha v$
  \item Un vettore $V \in V$ si chiama autovettore di $f$ se
$v \neq 0$ e $\exists \alpha \in \mathbb{K}$ t.c. $f\left(v\right) = \alpha v$
\end{enumerate}

\subparagraph{Oss} Diagonalizzare $f$ equivale a trovare una base formata di
autovettori.

\subparagraph{Oss} $0$ autovalore $\Leftrightarrow ker\left(f\right) \neq 0
\Leftrightarrow f$ non è un isomorfismo.

\paragraph{Teorema} Sia $f: V \rightarrow V$ un endomorfismo. Siano $v_1, ...,
v_k \in V$ autovettori di $f$ corrispondenti ad autovalori distinti $\alpha_1,
..., \alpha_k$. Allora $v_1, ..., v_k$ sono linearmente indipendenti.

\subsection{Come trovare autovalori}
\paragraph{Oss} $\alpha \in \mathbb{K}$ autovalore $\Leftrightarrow \exists
v \in V, v \neq 0, t.c. f\left(v\right) = \alpha v \Leftrightarrow \exists v
\neq 0 t.c. \left(f-\alpha \cdot id_V\right)\left(v\right) = 0 \Leftrightarrow
ker\left(f-\alpha \cdot id_V\right) \neq 0 \Leftrightarrow det\left(f-\alpha
\cdot id_V\right) = 0$.

\paragraph{Prop} Sia $n = dim V \in \mathbb{N}$, allora $det\left(f -
x \cdot id_V\right)$ è un polinomio in $x$ di grado $n$. Inoltre
\[det\left(f - x \cdot id_V\right) = (-1)^n x^n + a_{n-1}x^{n-1} + ... + a_1 x
+ det\left(f\right)\]
dove $a_{n-1},...,a_1 \in \mathbb{K}$

\paragraph{Polinomio caratteristico} ${\cal
X}_f\left(x\right) = det\left(f-x\cdot id_V\right) \in
\mathbb{K}\left[x\right]$ è il polinomio caratteristico di $f$.\\
Per $A \in {\cal M}_{n,n}\left(\mathbb{K}\right)$ il polinomio caratteristico
di $A$ è ${\cal X}_A\left(x\right) = det\left(A - x \cdot I_n\right)$.

\subparagraph{Prop} $\alpha$ è autovalore di $f \Leftrightarrow
{\cal X}_f\left(\alpha\right) = 0$. ($\alpha$ è autovalore di $A \Leftrightarrow
{\cal X}_A\left(\alpha\right) = 0$).\\
$\Rightarrow$ Gli autovalori sono radici del polinomio caratteristico.

\paragraph{Molteplicità} Sia $f: V \rightarrow V$ endomorfismo, con $dim V \in
\mathbb{N}$ e sia $\alpha \in \mathbb{K}$ un autovalore di $f$.
\begin{enumerate}
    \item La molteplicità algebrica di $\alpha$ è il massimo $a \in \mathbb{N}$
      t.c. $\left(x - \alpha\right)^a$ sia divida ${\cal X}_f\left(x\right)$.
    \item La molteplicità geometrica di $\alpha$ è $g =
      dim\left(ker\left(f - \alpha \cdot id_V\right)\right)$ che per il teorema
      del rango è: $n-rg\left(f - \alpha \cdot id_V\right)$
    \item $ker\left(f - \alpha\cdot id_V\right) =
      \left\{\text{autovettori associati a }\alpha\right\} \cup
      \left\{0\right\}$ sè l'autospazio associato ad $\alpha$.
\end{enumerate}
\subparagraph{Prop} $1 \leq g \leq a$

\paragraph{Teorema} \[f \text{ è diagonalizzabile} \Leftrightarrow
\sum_{i=1}^{n}g_i = n\]
dove $g_1, ..., g_n$ sono le molteplicità geometriche degli autovalori di $f$.

\subparagraph{Cor} $f$ è diagonalizzabile se e solo se:
\begin{enumerate}
    \item ${\cal X}_f\left(x\right) = (-1)^n \prod\limits_{i=1}^{k}
      \left(x-\alpha_i\right)^{a_i}$, dove $\alpha_i \neq \alpha_j$ per $i \neq
      j$.
    \item $g_i = a_i \forall i = 1, ..., k$.
\end{enumerate}

\end{document}
